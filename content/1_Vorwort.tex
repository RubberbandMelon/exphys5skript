\thispagestyle{vorwort}
\section*{Vorwort}
Das Vorlesungsskript zur \emph{\LectureTitle} bei \LecturerTitle~\LecturerName~wurde im Auftrag des IHFG vorlesungsbegleitend von zwei Studierenden entwickelt. Fragen, Anregungen und Korrekturvorschlägen zum Skript dürfen gerne an \ContactEmail~gestellt werden. Wenn nicht anderweitig gekennzeichnet, beziehen sich alle Informationen auf die Vorlesung im \LectureSemester.\par

Anmerkungen, welche den Aufschrieb ergänzten, sind wie folgt gekennzeichnet:
\begin{verbal}
    Dies ist ein Beispiel, wie mündliche Anmerkungen während der Vorlesung dargestellt sind.
\end{verbal}

Definitionen und \quickdef{erstmals eingeführte Begriffe} sind blau markiert:
\begin{definition}{Beispieldefinition}
    Dies ist ein Beispiel, wie Definitionen dargestellt sind.
\end{definition}

Besonders wichtige Fakten und zusammenhänge sind rot gekennzeichnet:
\begin{important}
    \textbf{Informationen in roten Kästen sind besonders wichtig und merkenswert.}
\end{important}