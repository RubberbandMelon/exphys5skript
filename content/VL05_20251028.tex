\setcounter{section}{3}
\setcounter{subsection}{2}
\setcounter{equation}{20}

\subsection{Das Wasserstoffmolekül \ce{H2}}
\subsubsection{Das Variationsprinzip}
Die Schrödingergleichung lautet
\begin{equation}
	\label{3.21}
	H \psi = E \psi.
\end{equation}
Diese Gleichung mit $ \psi^{*}$ multipliziert und über alle Elektronenkoordinaten integriert ergibt
\begin{equation}
	E = \frac{ \int \psi^{*} \psi \mathrm{d} V_1 \ldots \mathrm{d} V_n}{ \int \psi^{*} \psi \mathrm{d} V_1 \ldots\mathrm{d}V_{n}}
\end{equation}
mit der Anzahl der Elektronen $n$ und dem Erwartungswert der Energie $E$, der mit dem Energieeigenwert der Schrödingergleichung \ref{3.21} identisch ist, falls $ \psi$ die Eigenfunktion ist. Ist $ \psi$ nicht die Eigenfunktion der Schrödingergleichung dann wird der Energieerwartungswert höher als der Eigenwert der Lösung sein. Damit ergibt sich ein Kriterium für eine gültige Näherung (Beweis im Atkins S183).
Einige Ansätze für die Bestimmung der Wellenfunktionen der Elektronen im \ce{H2} Molekül sind
 \begin{itemize}
	\item Heitler London
	\item Hund-Mulliken-Bloch (LCAO) 
	\item Kovalent-Ionische Resonanz
\end{itemize}
welche alle von physikalischer Intuition getragen werden.

\subsubsection{Heitler-London-Methode für \ce{H2} (Valence-Bond-Methode)}
\begin{figure}[H]
    \centering
    \incfig{vl05_abb1_kerne_u_elek}
    \label{fig:vl05_abb1_kerne_u_elek}
\end{figure}

\begin{equation}
	\label{3.23}
	\begin{split}
		H &= \underbrace{- \frac{\hbar ^2}{2 m_0} \Delta_1 - \frac{\mathrm{e}^2}{4 \pi \varepsilon_0 r_{a_1}}}_{H_1}  \underbrace{- \frac{\hbar ^2}{2 m_0} \Delta_2 - \frac{\mathrm{e}^2}{4 \pi \varepsilon_0 r_{b_2}}}_{H_2}\\
	  &= \frac{\mathrm{e}^2}{4 \pi \varepsilon_0 r_{b1}} - \frac{\mathrm{e}^2}{4 \pi \varepsilon_0 r_{a 2}} + \frac{\mathrm{e}^2}{4 \pi \varepsilon_0 R_{ab}}+ \frac{\mathrm{e}^2}{4 \pi \varepsilon_0 r_{12}} \\
	\end{split}
\end{equation}
unter der Annahme $m_{\text{Kern}} \to  \infty$. Wir lösen jetzt die Schrödignergleichung
\begin{equation}
	\label{3.24}
	H \psi\left( \underline{r}_{1}, \underline{r}_{2} \right) = E \psi \left( \underline{r}_{1}, \underline{r}_{2} \right).
\end{equation} 

\begin{enumerate}
	\item Wir betrachten die Kerne als $\infty$-weit voneinander entfernt. Es folgt
		\begin{align}
			\label{3.25}
			H_1 \varphi_{a} \left( \underline{r}_{1} \right) &= E_0 \varphi_{a} \left( \underline{r}_{1} \right)  \\
			\label{3.26}
			H_2 \varphi_{b}\left( \underline{r}_{2} \right) &= E_0 \varphi_{b} \left( \underline{r}_{2} \right)  
		\end{align}
		Auch $ \varphi = \varphi_{a}\left( \underline{r}_{1} \right) \varphi_{b} \left( \underline{r}_{2} \right) $ ist eine Lösung der Schrödingergleichung mit dem Hamiltonoperator $H_1 +H_2$.\\
Bisher wurde der Spin vernachlässigt, jedoch muss nach dem Pauli-Prinzip die Gesamtwellenfunktion antisymmetrisch sein.\\
Funktion für Spin nach dem $\uparrow$ für $e_{1}$: $ \alpha \left( 1 \right) $, $ \beta\left( 1 \right) \equiv \downarrow$ \\
Funktion für Spin nach dem $\downarrow$ für $e_{2}$: $ \beta \left( 2 \right) $, $ \alpha \left( 2 \right) \equiv \uparrow$\\
fehlt was\\
Diese Wellenfunktion genügt aber nicht dem Pauliprinzip\\
\item Wir führen eine neue Wellenfunktion 
\begin{equation}
	\label{3.28}
	\psi = \varphi_{a} \left( \underline{r}_{1} \right) \alpha\left( 1 \right) \varphi_{b} \left( \underline{r}_{2} \right) \alpha\left( 2 \right) - \varphi_{a} \left( \underline{r}_{2} \right) \alpha\left( 2 \right) \varphi_{b} \left( \underline{r}_{1} \right) \alpha\left( 1 \right) 
\end{equation}
ein, welche antisymmetrisch bezüglich Vertauschungsoperation $ 1 \leftrightarrow 2$ der Ortswellenfunktion und Spin-Wellenfunktion ist.\\
Umformen: Schreibe \ref{3.28}
\begin{equation}
	\label{3.29}
	\psi = \alpha \left( 1 \right) \alpha \left( 2 \right) \left[ \varphi_{a} \left( \underline{r}_{1} \right) \varphi_{b} \left( \underline{r}_{2} \right) - \varphi_{a} \left( \underline{r}_{2}  \right) \varphi_{b}\left( \underline{r}_{1} \right)  \right] 
\end{equation}
wobei das Produkt der $ \alpha$ der symmetrischen Spin-WF und der Inhalt der Inhalt der Klammer der antisymmetrischen Orts-WF $ \psi_{\text{u}}$ entspricht.
Wir können \ref{3.28} im Hinblick auf Mehrelektronensystem als Determinante schreiben
$$
D = \begin{vmatrix}
	\varphi_{a}\left( \underline{r}_{1} \right)  \alpha \left( 1 \right)  & \varphi_{a} \left( \underline{r}_{2} \right) \alpha\left( 2 \right) \\
	\varphi_{b}\left( \underline{r}_{1} \right) \alpha\left( 1 \right) & \varphi_{b} \left( \underline{r}_{2} \right) \alpha\left( 2 \right) 
\end{vmatrix}
$$ 
mit $D = \text{Produkt der Haupdiagonale}-\text{Produkt der Nebendiagonale}$.\\
Konstruire Gesamt-Wellenfunktion mit parallelen Spins nach unten $\downarrow \downarrow$ 
\begin{equation}
	\label{3.32}
	\psi = \beta\left( 1 \right) \beta\left( 2 \right) \psi_{u}
\end{equation}

Es gibt noch eine weitere Wellenfunktion, die eine ungerade Orts-Wellenfunktion hat 
$$
\psi = \underbrace{\left[ \alpha\left( 1 \right) \beta\left( 2 \right)  + \alpha\left( 2 \right) \beta\left( 1 \right)  \right]}_{z\text{-Komp. des Gesamtspins}=0} \psi_{n}
$$ 
Konstruire symmetrische Orts-Wellenfunktion mit antisymmetrischer Spin-Wellenfunktion z.B.
\begin{equation}
	\label{3.34}
	\psi = \varphi_{a} \left( \underline{r}_{1} \right) \varphi_{b} \left( \underline{r}_{2} \right) \alpha\left( 1 \right) \beta\left( 2 \right) \quad \text{\textbf{nicht} antisymmetrisch}
\end{equation}
Besser in Analogie zu \ref{3.29}
 \begin{equation}
	 \label{3.35}
	 \psi = \underbrace{\left[ \varphi_{a} \left( \underline{r}_{1} \right) \varphi_{b}\left( \underline{r}_{2} \right) + \varphi_{a}\left( \underline{r}_{2} \right) \varphi_{b}\left( \underline{r}_{1} \right)  \right]}_{ \psi_{\text{g}}} \left[ \alpha\left( 1 \right) \beta\left( 2 \right) - \alpha\left( 2 \right) \beta\left( 1 \right)  \right] 
\end{equation}
\item Im Hamiltonoperator kommen keine Operatoren vor, die auf den Spin wirken\\
$\to $ Spinfunktionen können fürs weitere vernachlässigt werden.\\
Bemerkung: Wir vernachlässigen hier also Spin-Spin und Spin-Bahn-Wechselwirkung!\\

\end{enumerate}


Benutze $ \psi_{\text{u}}$ und $ \psi_{\text{g}}$ in Schrödingergleichung \ref{3.24} und berechne Energiewechselwirkungswerte nach dem Variationsprinzip \ref{3.24} (Die volle Rechnung ist im Hagen-Wolf auf S.63-67 zu finden)\\
Neue Integrale, die bei \ce{H_2^{+}}-Problem nicht auftauchen, sind
\begin{enumerate}
	\item $$
		\iint \varphi_{a}^2\left( \underline{r}_{1} \right) \varphi_{b}^{2} \left( \underline{r}_{2} \right) \frac{ \mathrm{e}^2}{4 \pi \varepsilon_0 r_{12}} \mathrm{d} V_1 \mathrm{d} V_2 = E_{\text{WW}}
	$$ 
	$\to $ Abstoßende Coulomb-Wechselwirkungsenergie der beiden $\mathrm{e}^{-}$-Wolken
\item 
	$$
	\iint \varphi_{b} \left( \underline{r}_{1} \right) \varphi_{a} \left( \underline{r}_{2} \right) \frac{\mathrm{e}^2}{4 \pi \varepsilon_0 r_{12}} \varphi_{a} \left( \underline{r}_{1} \right) \varphi_{b}\left( \underline{r}_{2} \right) \mathrm{d} V_1 \mathrm{d} V_2 = E_{\text{AW}}
	$$ 
	$\to $ stellt die Coulomb'sche-Wechselwirkung zwischen den Elektronen dar, wobei nicht die normale Ladungsdichte auftritt sondern die Austauschdichte \\ 
	$\to $ Coulomb'sche Austausch-Wechselwirkung
\end{enumerate}

Endergebnis für $ \psi_{\text{g}}$ und $ \psi_{\text{u}}$:
\begin{align}
	\label{3.51}
	E_{g} &= 2E_0 + \frac{2C + E_{\text{WW}}}{1 + S^2} + \frac{2DS + E_{\text{AW}}}{1 + S^2} + \frac{\mathrm{e}^2}{4 \pi \varepsilon_0 R_{a,b}} \\
	\label{3.52}
	E_{u} &= 2E_0 + \frac{2C + E_{WW}}{1- S^2}- \frac{2DS + E_{\text{AW}}}{1-S^2} + \frac{\mathrm{e}^2}{4 \pi \varepsilon_{0} R_{a,b}} 
\end{align}
Nur wenn die Energie tiefer liegt als $2 E_0$, was unendlich weit entfernt liegen Kernen entspricht, kommt es zu einer chemischen Bindung.\\
Es konkurieren verschiedene Effekte 
\begin{itemize}
	\item $C$ ist negativ, $E_{\text{WW}}$ und Kernabstoßung positiv
	\item Typ. QM-Effekte
		$$
		K = \underbrace{2DS}_{<0} + \underbrace{E_{\text{AW}}}_{>0} 
		$$ 
		Integrale müssen numerisch gelöst werden $\to $ $K < 0$
 $$
\implies E\left( \psi_{g} \right) < E\left( \psi_{u} \right) 
$$ 
Außerdem zeigt sich, dass $E \left( \psi_{g} \right) < 2 E_0 \implies$ bindend
\begin{figure}[H]
    \centering
    \incfig{vl05_abb2_bind_abind}
    \label{fig:vl05_abb2_bind_abind}
\end{figure}
\end{itemize}
Die Energieabsenkung kommt neben den Austauscheffekten dadurch zustande, dass die Elektronen, wie schon bei $\ce{H2+}$, sich gleichzeitig zwischen den Kernen aufhalten können, und so von dem Coulomb'schen Anziehungspotential beider Kerne profitieren.\\
$E_{\text{Bdg}}$ ist nach Heitler-London-Rechnung \SI{3,14}{\electronvolt}, nach Messung hingegen \SI{4,48}{\electronvolt}.\\
Die Heitler-London Rechnung kann also die Bindung zumindest erklären, muss aber noch quantitativ verbessert werden.

\subsubsection{Kovalent-Ionische Resonanz}
Bisher hatten wir angenommen, dass das eine Elektron jeweils gerade am anderen Kernort ist als das Andere.\\
%\begin{figure}[H]
%    \centering
%    \incfig{vl05_abb3_kernort}
%    \label{fig:vl05_abb3_kernort}
%\end{figure}
\begin{equation}
	\label{3.54}
	\psi_{\text{cov}} = N\left[ \varphi_{a} \left( \underline{r}_{1} \right) \varphi_{b} \left( \underline{r}_{2} \right) + \varphi_{a} \left( \underline{r}_{2} \right) \varphi_{b} \left( \underline{r}_{1} \right)  \right] 
\end{equation} 
Auch möglich\\
\textcolor{red}{Grafik}
Die Wellenfunktion wäre dann
\begin{align}
	\label{3.55}
	\psi&= \varphi_{a} \left( \underline{r}_{1} \right) \varphi_{a}\left( \underline{r}_{2} \right) \\
	\psi&= \varphi_{b}\left( \underline{r}_{1} \right) \varphi_{b}\left( \underline{r}_{2} \right) 
\end{align}

In symmetrischer Form:
\begin{equation}
	\psi_{\text{ion}} = N' \left[ \varphi_{a}\left( \underline{r}_{1} \right) \varphi_{a}\left( \underline{r}_{2} \right) + \varphi_{b} \left( \underline{r}_{1} \right) \varphi_{b} \left( \underline{r}_{2} \right)  \right] 
\end{equation}

Idealerweise verwendet man beide Möglichkeiten als Linearkombination
\begin{equation}
	\label{3.58}
	\psi = \psi_{\text{cov}} + c \psi_{\text{ion}} 
\end{equation}
wobei $c$ so zu bestimmen ist, dass der Eigenwert der Energie minimal wird $\left( E_{B} \sim \SI{3,4}{\electronvolt} \text{ bis } \SI{3,5}{\electronvolt} \right) $
